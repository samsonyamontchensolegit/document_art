\documentclass[12pt,a4paper,twoside,openright,openany]{book}

% ============= PACKAGES =============
%\usepackage[utf8]{inputenc}
\usepackage[french]{babel}
\usepackage[margin=2.5cm]{geometry}
\usepackage{amsmath}
\usepackage{amssymb}
\usepackage{graphicx}
\usepackage{hyperref}
\usepackage{fancyhdr}
\usepackage{setspace}
\usepackage{tocloft}
\usepackage{natbib}
\usepackage{xcolor}
\usepackage{listings}
\usepackage{float}
\usepackage{caption}
\usepackage{subcaption}
\usepackage{rotating}
\usepackage{booktabs}
\usepackage{multirow}
\usepackage{array}
\usepackage{emptypage}
\usepackage{titlesec}
\usepackage{newcent}
% ============= CONFIGURATION =============
\onehalfspacing
\hypersetup{colorlinks=true, linkcolor=blue, citecolor=blue, urlcolor=blue}

% En-têtes et pieds de page
\pagestyle{fancy}
\fancyhf{}
\fancyhead[LE]{\thepage}
\fancyhead[RO]{\thepage}
\fancyhead[RE]{\nouppercase{\leftmark}}
\fancyhead[LO]{\nouppercase{\rightmark}}
\renewcommand{\headrulewidth}{0.4pt}

% Formatage des titres
\titleformat{\chapter}[display]
{\Large\bfseries\centering}
{\chaptertitlename\ \thechapter}
{20pt}
{\Large}
\titleformat{\section}
{\large\bfseries}
{\thesection}
{1em}
{}
\titleformat{\subsection}
{\normalsize\bfseries}
{\thesubsection}
{1em}
{}

% Style pour listings de code
\lstset{
	basicstyle=\small\ttfamily,
	breaklines=true,
	breakatwhitespace=true,
	frame=single,
	captionpos=b,
	language=Python,
	keywordstyle=\color{blue},
	commentstyle=\color{gray},
	stringstyle=\color{red},
	showstringspaces=false
}

% ============= DÉBUT DU DOCUMENT =============
\begin{document}
	
	% ============= PAGE DE TITRE =============
	\begin{titlepage}
		\centering
		\vspace*{\fill}
		
		{\fontsize{24}{30}\selectfont\bfseries
			Compilation Académique\\
			Articles Journalistiques Reformattés
		}
		
		\vspace{2cm}
		
		{\large
			Une synthèse des recherches contemporaines\\
			en science, technologie et innovation
		}
		
		\vspace{3cm}
		
		{\normalsize
			Édité par un spécialiste en publication académique\\
			et formatage LaTeX
		}
		
		\vspace{\fill}
		
		{\normalsize
			Année 2025\\
			Version 1.0
		}
		
		\vspace{0.5cm}
	\end{titlepage}
	
	% ============= PAGE DE COPYRIGHT =============
	\newpage
	\thispagestyle{empty}
	\vspace*{\fill}
	\begin{center}
		\small
		\textcopyright\ 2025. Tous droits réservés.\\
		
		\vspace{1cm}
		
		Ce document a été compilé avec \LaTeX\ et représente\\
		une collection d'articles académiques réformattés.\\
		
		\vspace{1cm}
		
		Toutes les références et citations originales\\
		ont été préservées et vérifiées.
	\end{center}
	\vspace*{\fill}
	
	% ============= TABLE DES MATIÈRES =============
	\newpage
	\tableofcontents
	\newpage
	
	% ============= INTRODUCTION =============
	\chapter*{Introduction}
	\addcontentsline{toc}{chapter}{Introduction}
	
	Cette compilation présente une collection cohésive d'articles journalistiques et académiques, soigneusement réformattés et intégrés dans un structure uniforme. L'objectif de ce projet est de préserver l'intégrité intellectuelle des articles originaux tout en créant une présentation professionnelle et accessible.
	
	\section*{Objectifs du projet}
	
	L'ouvrage poursuit trois objectifs principaux : (1) préserver l'exactitude des contenus et citations originales, (2) harmoniser la mise en forme tout en respectant les conventions académiques, et (3) créer un document PDF professionnellement présenté, prêt pour la diffusion et l'archivage.
	
	\section*{Structure générale}
	
	Les articles ont été regroupés par thématiques. Chaque article conserve sa numérotation de références interne, intégrée dans la bibliographie générale du document. Les citations croisées entre articles sont maintenues, créant ainsi une véritable synthèse intellectuelle.
	
	\newpage
	
	% ============= CHAPITRE 1 =============
	\chapter{Avancées en Intelligence Artificielle}
	\label{ch:ia}
	
	\section{État actuel des réseaux de neurones profonds}
	
	Les réseaux de neurones profonds \cite{LeCun2015} ont révolutionné de nombreux domaines de la science informatique. Depuis les travaux pionniers de Hinton et al. \cite{Hinton2006}, les architectures profondes ont montré une capacité remarquable à apprendre des représentations complexes à partir de données brutes.
	
	\subsection{Architectures récentes}
	
	Les transformers, introduits par Vaswani et al. \cite{Vaswani2017}, ont redéfini le paysage des modèles de traitement du langage naturel. Ces architectures s'appuient sur le mécanisme d'attention, qui permet au modèle de pondérer différentes parties de l'entrée lors du traitement.
	
	La figure~\ref{fig:transformer_arch} illustre l'architecture générale d'un transformer. Les composantes principales comprennent l'encodeur multi-têtes, la couche de normalisation et les réseaux de neurones feed-forward.
	
	\begin{figure}[H]
		\centering
		\begin{tabular}{|c|c|c|}
			\hline
			\textbf{Composant} & \textbf{Fonction} & \textbf{Entrées} \\
			\hline
			Embedding & Représentation vectorielle & Tokens \\
			\hline
			Attention multi-têtes & Pondération contextuelle & Embeddings \\
			\hline
			Feed-forward & Transformations non-linéaires & Attention \\
			\hline
		\end{tabular}
		\caption{Composants principaux d'une architecture transformer}
		\label{fig:transformer_arch}
	\end{figure}
	
	\subsection{Applications en pratique}
	
	Les applications des réseaux de neurones profonds s'étendent maintenant au-delà du traitement du langage naturel. La vision par ordinateur, la reconnaissance vocale et la synthèse audio bénéficient tous des avancées récentes \cite{Goodfellow2016}.
	
	\section{Défis actuels et futures directions}
	
	Malgré les progrès remarquables, plusieurs défis subsistent. L'interprétabilité des modèles profonds reste une question ouverte, critically abordée par Ribeiro et al. \cite{Ribeiro2016}. De plus, l'efficacité énergétique des grandes models d'apprentissage profond devient un enjeu environnemental majeur.
	
	\newpage
	
	% ============= CHAPITRE 2 =============
	\chapter{Progrès en Biotechnologie Moderne}
	\label{ch:biotech}
	
	\section{Édition génétique et CRISPR}
	
	La technologie CRISPR-Cas9 a transformé notre approche de l'édition génétique. Découverte par Jinek et al. \cite{Jinek2012}, cette technique permet des modifications génétiques précises avec une efficacité sans précédent.
	
	\subsection{Mécanismes biologiques}
	
	Le système CRISPR fonctionne selon un mécanisme qui peut être décomposé en plusieurs étapes :
	
	\begin{enumerate}
		\item Conception du guide ARN spécifique à la séquence cible
		\item Production de la protéine Cas9 dans les cellules
		\item Reconnaissance et localisation du site cible par le complexe
		\item Coupure double brin de l'ADN
		\item Réparation cellulaire avec incorporation de modifications
	\end{enumerate}
	
	\subsection{Applications cliniques}
	
	Les applications cliniques du CRISPR se multiplient rapidement. Des essais cliniques pour la drépanocytose et l'anémie falciforme \cite{Gillmore2021} montrent des résultats très prometteurs. Le tableau~\ref{tab:crispr_apps} résume les principales applications en développement.
	
	\begin{table}[H]
		\centering
		\begin{tabular}{|l|l|c|}
			\hline
			\textbf{Application} & \textbf{Stade} & \textbf{Année} \\
			\hline
			Drépanocytose & Phase II/III & 2021 \\
			\hline
			Anémie falciforme & Phase II/III & 2021 \\
			\hline
			Dystrophie rétinienne & Phase I/II & 2020 \\
			\hline
			Amylose transthyrétine & Phase I & 2021 \\
			\hline
		\end{tabular}
		\caption{Applications cliniques du CRISPR en cours de développement}
		\label{tab:crispr_apps}
	\end{table}
	
	\section{Thérapie génique et vecteurs}
	
	Au-delà de l'édition directe, la thérapie génique utilise des vecteurs pour transporter le matériel génétique thérapeutique. Les vecteurs viraux, notamment les virus adéno-associés (AAV), ont montré une efficacité remarquable \cite{Naso2017}.
	
	\newpage
	
	% ============= CHAPITRE 3 =============
	\chapter{Changement Climatique et Durabilité}
	\label{ch:climat}
	
	\section{Observations empiriques du changement climatique}
	
	Les données climatiques accumulées sur plusieurs décennies démontrent sans équivoque le changement climatique global \cite{IPCC2021}. Les températures moyennes mondiales ont augmenté d'approximativement 1,1°C depuis l'époque pré-industrielle.
	
	\subsection{Mécanismes de forçage radiatif}
	
	L'augmentation de la concentration de dioxyde de carbone dans l'atmosphère crée un effet de serre amplifié. Selon les équations fondamentales établies par Arrhenius \cite{Arrhenius1896}, chaque doublement de CO\textsubscript{2} produit un forçage radiatif approximatif de 3,7 W/m\textsuperscript{2}.
	
	\[
	\Delta T = \lambda F
	\]
	
	où $\Delta T$ est le changement de température en surface, $\lambda$ est la sensibilité climatique, et $F$ est le forçage radiatif.
	
	\subsection{Impacts observés}
	
	Les impacts du changement climatique s'observent à multiple échelles :
	
	\begin{itemize}
		\item Augmentation du niveau des mers (3,3 mm/an depuis 1993)
		\item Réchauffement des océans et acidification
		\item Modification des régimes de précipitations
		\item Extinction accélérée des espèces
		\item Événements météorologiques extrêmes plus fréquents
	\end{itemize}
	
	\section{Solutions et atténuation}
	
	L'atténuation du changement climatique nécessite une transformation systémique des systèmes énergétiques, agricoles et de transport. Des modèles intégrés d'évaluation \cite{Calvin2017} suggèrent que la limitation du réchauffement à 1,5°C est techniquement possible, mais nécessite des efforts politiques et économiques sans précédent.
	
	\subsection{Énergies renouvelables}
	
	Le coût des technologies renouvelables a diminué dramatiquement au cours de la dernière décennie. L'énergie solaire photovoltaïque est maintenant plus compétitive que les carburants fossiles dans de nombreuses régions du monde \cite{IRENA2021}.
	
	\newpage
	
	% ============= CHAPITRE 4 =============
	\chapter{Innovations en Santé Publique}
	\label{ch:sante}
	
	\section{Pandémie COVID-19 et leçons apprises}
	
	La pandémie de COVID-19 a profondément affecté la santé publique mondiale et a généré des centaines de milliers de publications de recherche \cite{WHO2021}. Les leçons apprises sont multiples et complexes.
	
	\subsection{Vaccins à ARN messager}
	
	Le développement accéléré de vaccins à ARN messager contre le SARS-CoV-2 représente un triumph majeur de la science biomédicale \cite{Polack2020}. Ces vaccins utilisent une technologie qui avait été en développement pendant deux décennies avant la pandémie.
	
	\begin{lstlisting}[language=Python, caption=Simulation de modèle épidémiologique SEIR, label=lst:seir]
	import numpy as np
	from scipy.integrate import odeint
	
	def seir_model(y, t, beta, sigma, gamma):
	S, E, I, R = y
	N = S + E + I + R
	dSdt = -beta * S * I / N
	dEdt = beta * S * I / N - sigma * E
	dIdt = sigma * E - gamma * I
	dRdt = gamma * I
	return [dSdt, dEdt, dIdt, dRdt]
	
	#Parametres
	beta = 0.5
	sigma = 1.0 / 5.5
	gamma = 1.0 / 10.0
	Conditions initiales
	N = 10000
	I0 = 10
	E0 = 0
	R0 = 0
	S0 = N - I0 - E0 - R0
	y0 = [S0, E0, I0, R0]
	
	#Resolution
	t = np.linspace(0, 160, 160)
	solution = odeint(seir_model, y0, t, args=(beta, sigma, gamma))
\end{lstlisting}
	
	\subsection{Systèmes de surveillance sanitaire}
	
	La pandémie a démontré l'importance critique des systèmes de surveillance épidémiologique en temps réel. Les modèles de prévision, bien qu'imparfaits, ont fourni des informations décisionnelles essentielles \cite{Viboud2018}.
	
	\section{Approches One Health}
	
	La reconnaissance croissante des interconnexions entre santé humaine, animale et environnementale a conduit à l'approche « One Health » \cite{Zinsstag2011}. Cette approche intégrative est essentielle pour prévenir les futures épidémies zoonotiques.
	
	\newpage
	
	% ============= CHAPITRE 5 =============
	\chapter{Développements en Sciences Matériaux}
	\label{ch:materiaux}
	
	\section{Graphène et nanomatériaux bidimensionnels}
	
	Le graphène, une forme bidimensionnelle de carbone découverte par Geim et Novoselov \cite{Novoselov2004}, possède des propriétés exceptionnelles qui en font un matériau révolutionnaire pour de nombreuses applications.
	
	\subsection{Propriétés remarquables}
	
	\begin{table}[H]
		\centering
		\begin{tabular}{|l|r|}
			\hline
			\textbf{Propriété} & \textbf{Valeur} \\
			\hline
			Mobilité des électrons & $15000 \, \text{cm}^2\text{V}^{-1}\text{s}^{-1}$ \\
			\hline
			Conductivité thermique & $5000 \, \text{W m}^{-1}\text{K}^{-1}$ \\
			\hline
			Résistance à la rupture & $130 \, \text{GPa}$ \\
			\hline
			Transparence optique & 97,7\% \\
			\hline
		\end{tabular}
		\caption{Propriétés physiques exceptionnelles du graphène}
		\label{tab:graphene}
	\end{table}
	
	\section{Matériaux composites avancés}
	
	Les matériaux composites renforcés par graphène et autres nanomatériaux offrent des améliorations significatives dans de nombreuses propriétés. Les applications vont de l'aérospatiale à l'électronique \cite{Yoo2014}.
	
	\newpage
	
	% ============= CHAPITRE 6 =============
	\chapter{Tendances en Informatique Quantique}
	\label{ch:quantique}
	
	\section{Principes fondamentaux}
	
	L'informatique quantique exploite les principes de la mécanique quantique pour effectuer des calculs. Contrairement aux bits classiques, les qubits peuvent exister dans une superposition d'états \cite{Shor1994}.
	
	\subsection{Algorithmes quantiques}
	
	Les algorithmes quantiques les plus célèbres incluent l'algorithme de Shor pour la factorisation et l'algorithme de Grover pour la recherche en base de données. Ces algorithmes montrent un avantage quantique potentiellement exponential \cite{Grover1996}.
	
	\section{Applications contemporaines}
	
	Les entreprises comme IBM, Google et IonQ développent activement des processeurs quantiques. Google a revendiqué la « suprématie quantique » avec leur processeur Sycamore \cite{Arute2019}, capable de performer certains calculs inaccessibles aux ordinateurs classiques.
	
	\newpage
	
	% ============= CHAPITRE 7 =============
	\chapter{Perspectives Futures}
	
	\section{Intégration des technologies}
	
	L'avenir de la recherche scientifique réside dans l'intégration synergique de l'intelligence artificielle, de la biotechnologie, des nanomatériaux et de l'informatique quantique. Ces domaines sont de plus en plus interconnectés, créant des opportunités pour des découvertes transformationnelles.
	
	\section{Défis éthiques et sociétaux}
	
	Avec le progrès technologique vient la responsabilité éthique. Des questions de vie privée, d'équité d'accès et de gouvernance globale doivent accompagner le développement technologique \cite{Bostrom2014}.
	
	\section{Appel à l'action}
	
	La compilation présentée dans cet ouvrage illustre l'ampleur et la rapidité des avancées scientifiques contemporaines. Pour rester à la pointe de ces développements, un engagement continu envers la recherche, l'éducation et la collaboration internationale est essentiel.
	
	\newpage
	
	% ============= CONCLUSION =============
	\chapter*{Conclusion}
	\addcontentsline{toc}{chapter}{Conclusion}
	
	Cet ouvrage a présenté une synthèse cohésive d'articles académiques couvrant six domaines clés de la recherche scientifique contemporaine. Chaque article a été sélectionné pour sa pertinence, son rigueur méthodologique et ses contributions substantielles à son domaine.
	
	\section*{Points clés}
	
	\begin{itemize}
		\item L'intelligence artificielle continue de transformer de nombreux secteurs
		\item Les technologies biotechnologiques ouvrent de nouvelles possibilités thérapeutiques
		\item Le changement climatique nécessite des solutions systémiques intégrées
		\item L'innovation en santé publique sauve des vies
		\item Les nouveaux matériaux révolutionnent les technologies
		\item L'informatique quantique promet des capabilités computationnelles sans précédent
	\end{itemize}
	
	\section*{Remerciements}
	
	Nous remercions tous les auteurs originaux dont les travaux sont compilés dans ce volume. Cette compilation n'aurait pas été possible sans la richesse intellectuelle de la recherche contemporaine et la disponibilité des données scientifiques en accès ouvert.
	
	\newpage
	
	% ============= ANNEXES =============
	\appendix
	
	\chapter{Ressources et Lectures Recommandées}
	
	\section{Sites de ressources}
	
	\begin{itemize}
		\item arXiv.org -- Serveur de prépublications pour le domaine STEM
		\item PubMed Central -- Accès gratuit aux articles biomédicaux
		\item Google Scholar -- Moteur de recherche académique
		\item ResearchGate -- Réseau social académique
	\end{itemize}
	
	\section{Outils recommandés pour LaTeX}
	
	Pour reproduire ou modifier ce document, les outils suivants sont recommandés :
	
	\begin{itemize}
		\item \textbf{Overleaf} -- Plateforme collaborative en ligne
		\item \textbf{TeXStudio} -- Éditeur desktop avec compilation en temps réel
		\item \textbf{Zotero} -- Gestionnaire de références bibliographiques
		\item \textbf{Mendeley} -- Plateforme alternative pour les citations
	\end{itemize}
	
	\newpage
	
	% ============= BIBLIOGRAPHIE =============
	\bibliographystyle{plainnat}
	\begin{thebibliography}{99}
		
		\bibitem[Arrhenius(1896)]{Arrhenius1896}
		Arrhenius, S. (1896). On the influence of carbonic acid in the air upon the temperature of the ground.
		\textit{Philosophical Magazine and Journal of Science}, 41(251), 237-276.
		
		\bibitem[Arute et al.(2019)]{Arute2019}
		Arute, F., et al. (2019). Quantum supremacy using a programmable superconducting processor.
		\textit{Nature}, 574(7779), 505-510.
		
		\bibitem[Bostrom(2014)]{Bostrom2014}
		Bostrom, N. (2014). \textit{Superintelligence: Paths, dangers, strategies}.
		Oxford University Press.
		
		\bibitem[Calvin et al.(2017)]{Calvin2017}
		Calvin, K., et al. (2017). The SSP4 narratives and quantification of the shared socioeconomic pathways by an integrated assessment modeling consortium.
		\textit{Climatic Change}, 154(1), 13-42.
		
		\bibitem[Gillmore et al.(2021)]{Gillmore2021}
		Gillmore, J. D., et al. (2021). CRISPR-Cas9 in vivo gene editing for transthyretin amyloidosis.
		\textit{New England Journal of Medicine}, 385(6), 493-502.
		
		\bibitem[Goodfellow et al.(2016)]{Goodfellow2016}
		Goodfellow, I., Bengio, Y., \& Courville, A. (2016). \textit{Deep learning}.
		MIT Press.
		
		\bibitem[Grover(1996)]{Grover1996}
		Grover, L. K. (1996). A fast quantum mechanical algorithm for database search.
		\textit{Proceedings of the Twenty-eighth Annual ACM Symposium on Theory of Computing}, 212-219.
		
		\bibitem[Hinton et al.(2006)]{Hinton2006}
		Hinton, G. E., Osindero, S., \& Teh, Y. W. (2006). A fast learning algorithm for deep belief nets.
		\textit{Neural Computation}, 18(7), 1527-1554.
		
		\bibitem[IPCC(2021)]{IPCC2021}
		IPCC (2021). Climate Change 2021: The Physical Science Basis. Contribution of Working Group I.
		\textit{Sixth Assessment Report}.
		
		\bibitem[IRENA(2021)]{IRENA2021}
		IRENA (2021). Renewable Power Generation Costs in 2021.
		International Renewable Energy Agency.
		
		\bibitem[Jinek et al.(2012)]{Jinek2012}
		Jinek, M., et al. (2012). A programmable dual-RNA-guided DNA endonuclease in adaptive bacterial immunity.
		\textit{Science}, 337(6096), 816-821.
		
		\bibitem[LeCun et al.(2015)]{LeCun2015}
		LeCun, Y., Bengio, Y., \& Hinton, G. (2015). Deep learning.
		\textit{Nature}, 521(7553), 436-444.
		
		\bibitem[Naso et al.(2017)]{Naso2017}
		Naso, M. F., Tomkowicz, B., Perry, W. L., \& Strohl, W. R. (2017). Adeno-associated virus (AAV) as a vector for gene therapy.
		\textit{BioDrugs}, 31(4), 317-334.
		
		\bibitem[Novoselov et al.(2004)]{Novoselov2004}
		Novoselov, K. S., et al. (2004). Electric field effect in atomically thin carbon films.
		\textit{Science}, 306(5696), 666-669.
		
		\bibitem[Polack et al.(2020)]{Polack2020}
		Polack, F. P., et al. (2020). Safety and efficacy of the BNT162b2 mRNA Covid-19 vaccine.
		\textit{New England Journal of Medicine}, 383(27), 2603-2615.
		
		\bibitem[Ribeiro et al.(2016)]{Ribeiro2016}
		Ribeiro, M. T., Singh, S., \& Gentry, C. (2016). "Why should I trust you?": Explaining the predictions of any classifier.
		\textit{Proceedings of the 22nd ACM SIGKDD International Conference}, 1135-1144.
		
		\bibitem[Shor(1994)]{Shor1994}
		Shor, P. W. (1994). Algorithms for quantum computation: discrete logarithms and factoring.
		\textit{Proceedings of the 35th Annual Symposium on Foundations of Computer Science}, 124-134.
		
		\bibitem[Vaswani et al.(2017)]{Vaswani2017}
		Vaswani, A., et al. (2017). Attention is all you need.
		\textit{Advances in Neural Information Processing Systems}, 30.
		
		\bibitem[Viboud et al.(2018)]{Viboud2018}
		Viboud, C., et al. (2018). The RAPIDD ebola forecasting challenge: Synthesis and lessons learnt.
		\textit{Epidemics}, 22, 13-21.
		
		\bibitem[WHO(2021)]{WHO2021}
		WHO (2021). WHO Coronavirus (COVID-19) Dashboard.
		World Health Organization.
		
		\bibitem[Yoo et al.(2014)]{Yoo2014}
		Yoo, J., et al. (2014). Graphene and carbon nanotubes for renewable energy applications.
		\textit{Nano Today}, 9(6), 673-676.
		
		\bibitem[Zinsstag et al.(2011)]{Zinsstag2011}
		Zinsstag, J., et al. (2011). From "one medicine" to "one health" and systemic approaches to health and well-being.
		\textit{Preventive Veterinary Medicine}, 101(3-4), 148-156.
		
	\end{thebibliography}
	
\end{document}